\documentclass[compress]{beamer}
\usetheme{Warsaw}
\usecolortheme{seagull}
\setbeamertemplate{headline}{}
\beamertemplatenavigationsymbolsempty
\useoutertheme{infolines}

\usepackage{lipsum}
\setbeamersize{text margin left=10pt,text margin right=10pt}
\setbeamertemplate{enumerate items}[default]
\setbeamertemplate{itemize items}[default]

%\addtobeamertemplate{navigation symbols}{}{%
%    \usebeamerfont{footline}%
%    \usebeamercolor[fg]{footline}%
%    \hspace{1em}%
%    \insertframenumber/\inserttotalframenumber
%}

\usepackage{graphicx}
\graphicspath{{/Users/jlochman/Desktop/Diploma-thesis/Chapter3/}}
\usepackage{epstopdf}
\usepackage{booktabs} 
\usepackage{courier}
\usepackage{color}
\usepackage{tikz}

\setbeamercovered{transparent}

\PassOptionsToPackage{demo}{graphicx}
\def\Put(#1,#2)#3{\leavevmode\makebox(0,0){\put(#1,#2){#3}}}

\newcommand{\GeV}{\,\text{GeV}}
\newcommand{\TeV}{\,\text{TeV}}
\newcommand{\pt}{p_{T}}

\title[High $\pt$ jets]{Hight $\pt$ jets in Run II of the ATLAS Experiment} 
\author{Jan Lochman}
\institute[FNSPE CTU] 
{
Czech Technical University \\ 
\medskip
\textit{jan.lochman@cern.ch} \\
\medskip
\medskip
ATLAS Meeting \\ 
\medskip
}
\date{\today}

\begin{document}

%------------------------------------------------

\begin{frame}
\titlepage 
\end{frame}

%------------------------------------------------

\section{Introduction}

\begin{frame}
\frametitle{My Analysis} 
\footnotesize

\begin{columns}[onlytextwidth]
  \begin{column}{0.5\textwidth}
    Inclusive jet double differential cross section in $\pt$ and rapidity
    $y$ (inclusive means $pp~\rightarrow$~jet~+~''anything'') in Run~II of the ATLAS
    Experiment ($\sqrt{s}=13\TeV$)
    \begin{figure}[b]
      \centering
      \includegraphics[width=\textwidth]{{../PrezentationATLASmeeting/ATLASinclusive04}.png}
    \end{figure}
  \end{column}
  \begin{column}{0.5\textwidth}
\begin{block}{Why Inclusive Jets?}
  \begin{itemize}
    \item They Cover a \textit{\color{red}wide range of momentum transfers}
      ($\sim 1 \GeV - 1 \TeV$ on the LHC) $\rightarrow$ predictions sensitive to
      the properties of the running coupling constant $\alpha_S$

    \item They probe the structure of proton at \textit{\color{red}small
    distance scales}
    \begin{equation*}
      \lambda \sim 1/\pt \sim \TeV^{-1} \sim 10^{-19}\,\text{m}
    \end{equation*}

    \item They contribute to our understanding of PDFs

    \item They \textit{\color{red}appreciate the increase in the center-of-mass
    energy} as no other physics process observed on hadron colliders
  \end{itemize}
\end{block}
  \end{column}
\end{columns}

\normalsize

\end{frame}

\begin{frame}
\frametitle{My Analysis} 

\begin{block}{Data}
  Monte Carlo generated events of $pp$ collisions at $\sqrt{s}=13\TeV$. 
  \begin{itemize}
    \item collisions generated with \textsc{Pythia8} (particle level)
    \item ATLAS detector response obtained with \textsc{Geant4} full
      simulation (detector level)
  \end{itemize}
\end{block}

\begin{block}{Jet Corrections}
  Cross section corrected from the detector to the particle level in two steps
  \begin{itemize}
    \item Calibration
    \item Unfolding
  \end{itemize}
\end{block}

\begin{block}{LO vs. NLO QCD}
  Particle level cross section from \textsc{Pythia8} (LO QCD) compared with the
  parton level NLO QCD cross section prediction.  
\end{block}

\end{frame}

\section{Jets}
\subsection{Introduction}

\begin{frame}
\frametitle{Why Do We Need Jets?}
\onslide<1->
\begin{columns}[onlytextwidth]
  \begin{column}{0.5\textwidth}
    \textbf{Gluon radiation cross section:}
    \textbf{Divergences:}
    \begin{itemize}
      \item \textit{\color{red}Infrared} ($E_k = 0$)
      \item \textit{\color{red}Collinear} ($\theta = 0$)
    \end{itemize}
    \textbf{Jet:} A group of collimated particles
  \end{column}
  \begin{column}{0.5\textwidth}
    \begin{equation*}
      \sigma_{q \rightarrow qg} \sim \frac{d\theta}{|\sin\theta|}
      \frac{dE_k}{E_k}
    \end{equation*}
    \begin{figure}[b]
      \centering
      \includegraphics[width=\textwidth]{{../PrezentationATLASmeeting/gluonRadiation}.png}
    \end{figure}
  \end{column}
\end{columns}
\onslide<2->
\textbf{Jet algorithm:} A prescription, how particles (or other objects) are clustered
  into separate jets. It should fulfill
\begin{columns}[onlytextwidth]
  \begin{column}{0.7\textwidth}
      \begin{itemize}
        \item \textit{Infrared safety:} The presence of an additional soft particle
          should not affect the recombination of particles into a jet.
        \item \textit{Collinear safety:} Jet reconstruction should not depend on the
          fact, if the energy is carried by one particle, of if the particle is
          split into more collinear particles.
      \end{itemize}
  \end{column}
  \begin{column}{0.3\textwidth}
    \begin{figure}[b]
      \centering
      \includegraphics[width=\textwidth]{{../PrezentationATLASmeeting/clustering}.png}
    \end{figure}
  \end{column}
\end{columns}
\textbf{Summary:} $q$ or $g$ {\color{red}CANNOT} be observed. Jets
{\color{red}CAN}.

\end{frame}


\subsection{Jet Algorithms}

\begin{frame}
\frametitle{Cone Jet Algorithms}
\begin{itemize}
  \item The {\color{red}most illustrative} jet algorithms. Different modifications.
  \item Used in Tevatron. Not used in ATLAS. 
\end{itemize}
\begin{columns}[onlytextwidth]
  \begin{column}{0.4\textwidth}
    \begin{figure}[t]
      \centering
      \includegraphics[width=0.9\textwidth]{{../Chapter2/IRsafety}.png}
      \\
      infrared unsafety
      \\
      \includegraphics[width=0.9\textwidth]{{../Chapter2/ColSafety}.png}
      \\
      collinear unsafety
    \end{figure}
  \end{column}
  \begin{column}{0.6\textwidth}
    \textbf{Algorithm:}
    \begin{enumerate}
      \item Take a particle with the highest $\pt > \pt^{cutoff}$
      \item Recombine all particles within the fixed cone 
      \item Update the cone direction
      \item If the direction has changed, go to 2, else you have a jet
      \item Go to 1 until there is no particle left with $\pt > \pt^{cutoff}$
    \end{enumerate}
  \end{column}
\end{columns}
\end{frame}

\begin{frame}
\frametitle{$k_t$ Jet Algorithms}
\textbf{Algorithm:}
\begin{enumerate}
  \onslide<1-> \item For each input object $i$ and all pairs of input objects $(i,j)$
    calculate
    \begin{equation*}
      d_i = p_{T,i}^{2p} \quad , \quad
      d_{ij} = \min \left( p_{T,i}^{2p}, p_{T,j}^{2p} \right)
      \frac{ \Delta y^2 + \Delta \phi^2 }{R^2} \quad \quad
      (R=0.4)
    \end{equation*}
    \begin{itemize}
      \item $p=1$ $k_t$ algorithm,
      \item $p=0$ Cambridge/Aachen algorithm,
      \item $p=-1$ {\color{red}anti-$k_t$ algorithm}.
    \end{itemize}
  \onslide<2-> \item Find minimum $d_{min}$ between all $d_{ij}$ and $d_i$
  \begin{itemize}
    \item $d_{min}$ is between $d_{ij}$'s.
      
      Recombine $i$, $j$ into a new object $k$. Remove $i$, $j$ from the list,
      add $k$ to the list.

    \item $d_{min}$ is between $d_i$'s.
      
      Object $i$ is a jet. Remove $i$ from the list.
  \end{itemize}
  \item Go to 1 until all input objects are part of a jet.
\end{enumerate}
\end{frame}

\begin{frame}
\frametitle{$k_t$ Jet Algorithms}
\begin{figure}[t]
  \centering
  \includegraphics[width=0.5\textwidth]{{../Chapter2/JetRecombination1}.png}
  \includegraphics[width=0.5\textwidth]{{../Chapter2/JetRecombination2}.png}
\end{figure}
\begin{itemize}
  \item $k_t$ jet algorithms are both infrared and collinear safe
  \item ATLAS uses anti-$k_t$ jet algorithm 
\end{itemize}
\end{frame}

\section{Jet Reconstruction}
\subsection{Parton, Particle and Detector Levels}

\begin{frame}
\frametitle{Jet Reconstruction}
Jet can be defined on a \textit{three different levels of collision}:
\begin{itemize}
  \item \textbf{Parton level} - quarks, gluons and other particles created just after the
    collision. Directly connected to the QCD processes.
  \item \textbf{Particle level} - particles created by the hadronization. 
  \item \textbf{Detector level} - recorded signal. Detector imperfections cause a
    \textit{\color{red}distortion of observables}.
\end{itemize}
\begin{figure}[b]
  \centering
  \includegraphics[width=0.7\textwidth]{{../Chapter2/JetPhases}.png}
\end{figure}
\end{frame}

\subsection{Jet Corrections}

\begin{frame}
\frametitle{Jet Corrections}
\begin{itemize}
  \item Correct observables derived from detector to particle level by
    removing the detector effects
  \item Two main procedures - \textit{\color{red}Calibration} and
    \textit{\color{red}Unfolding}
\end{itemize}
\begin{figure}[b]
  \centering
  \includegraphics[width=0.9\textwidth]{{../PrezentationATLASmeeting/JetCorrections}.png}
\end{figure}
\end{frame}

\subsection{Calibration}

\begin{frame}
\frametitle{Calibration}
\begin{itemize}
  \item Modifies the kinematic properties of \textit{\color{red}individual jets} - the most important
    correction: Energy
  \item Tries to minimize the calorimeter non-compensation, noise, losses in dead
    material and cracks, longitudinal leakage and particle deflection in magnetic
    field.
  \item \textit{\color{red}Universal} for each jet analysis. Uses the standard
    \textsc{ApplyJetCalibration} library.
\end{itemize}
\begin{figure}[b]
  \centering
  \includegraphics[width=0.9\textwidth]{{../PrezentationATLASmeeting/JetCalibration}.png}
\end{figure}
\end{frame}

\subsection{Unfolding}

\begin{frame}
\frametitle{Unfolding}
\begin{itemize}
  \item Corrects the observables from detector level, to observables on particle
    level. 
  \item Tries to minimize the effects of detector \textit{\color{red}finite resolution}.
  \item \textit{\color{red}Analysis dependent}.
\end{itemize}
\begin{figure}[b]
  \centering
  \includegraphics[width=0.39\textwidth]{{../PrezentationATLASmeeting/UnfoldingEffect}.png}
  \includegraphics[width=0.6\textwidth]{{SignalVSTruth}.eps}
\end{figure}
\end{frame}

\begin{frame}
\frametitle{Unfolding - Mathematical Formulation}
\begin{itemize}
  \onslide<1-> \item \textbf{I want:} $f(\pt)$ (distribution of inclusive jet $\pt$ for $\pt \in
    \langle a, b \rangle$)
  \item From detector level \textbf{I have:} $g(x)$ (distribution of unphysical variable
    $x$)
  \begin{equation*}
    g(x) = \int_a^b A(x,\pt) f(\pt) d\pt
  \end{equation*}
  \item Detector smearing described by $A(x,\pt)$
  \item Complicated \textit{\color{red}integral~equation} for $f(\pt)$
  \onslide<2-> \item Luckily $g(x)$ and $f(\pt)$ are for practical purpose discretized and in
    analysis, I assume $x \in \langle a, b \rangle$, $N(i) \subset \langle
    a , b \rangle$ 
  \begin{equation*}
    g_i = \int_{N(i)}g(x)dx \quad , \quad f_i = \int_{N(i)}f(\pt)d\pt
  \end{equation*}
  \item So the response of the detector is described by a ''simple''
    \textit{\color{red}matrix~equation}, with $A$ being called
    \textit{\color{red}Transfer Matrix} 
  \begin{equation*}
    g = A f
  \end{equation*}
\end{itemize}
\end{frame}

\section{Data Analysis}
\subsection{Data Characteristics}

\begin{frame}
\frametitle{Data Characteristics}
\begin{columns}[onlytextwidth]
  \begin{column}{0.4\textwidth}
\begin{itemize}
  \item $pp$ collisions at $\sqrt{s}=13\TeV$, anti-$k_t$ jet algorithm with
    $R=0.4$, CT10 PDFs, AU2
  \item Measuring of inclusive jet double differential cross section in $\pt$
    and rapidity $y$ 
\end{itemize}
  \end{column}
  \begin{column}{0.6\textwidth}
\begin{figure}[b]
  \centering
  \includegraphics[width=\textwidth]{{../PrezentationATLASmeeting/DataChar}.png}
\end{figure}
  \end{column}
\end{columns}
\begin{itemize}
  \item \textbf{Parton level} - cross section prediction calculated with NLOJET++ program
    (NLO~QCD)
  \item \textbf{Particle level} - events generated by \textsc{Pythia8} (LO~QCD)
  \item \textbf{Detector level} - detector response on \textsc{Pythia8} events obtained
    by \textsc{Geant4} full detector simulation. 
\end{itemize} 
\end{frame}

\begin{frame}
\frametitle{\textsc{Pythia8} Data Characteristics}
\begin{itemize}
  \item Events were generated in a slices according to the leading truth
      jet $\pt$.
  \item Slices differ in event weight which is for all event calculated as
  \begin{equation*}
    \text{weight} = \frac{\text{(Cross-section)} \cdot \text{(Filter
      Efficiency)} \cdot w_0}{\text{(\# events)}} 
    \end{equation*}
  \item $w_0$ is additional weight factor stored in \texttt{EventInfoAux}
    container
\end{itemize}
  \small
  \begin{table}
  \centering
  \begin{tabular}{|c|rcr|c|c|c|}
    \hline 
     JZ & \multicolumn{3}{|c|}{$\pt$ range (GeV)} & Cross-section (fb) & Filter Efficiency & \# events  \\ 
    \hline
    \hline
		 JZ0W &     0 & - &    20 & 7.8420e+13 & 9.7193e-01 & 3498000 \\ 
    \hline
		 JZ1W &    20 & - &    80 & 7.8420e+13 & 2.7903e-04 & 2998000 \\
    \hline
		 JZ2W &    80 & - &   200 & 5.7312e+10 & 5.2261e-03 & 500000  \\
    \hline
		 JZ3W &   200 & - &   500 & 1.4478e+09 & 1.8068e-03 & 499500  \\
    \hline
		 JZ4W &   500 & - &  1000 & 2.3093e+07 & 1.3276e-03 & 477000  \\
    \hline
		 JZ5W &  1000 & - &  1500 & 2.3793e+05 & 5.0449e-03 & 499000  \\
    \hline
		 JZ6W &  1500 & - &  2000 & 5.4279e+03 & 1.3886e-02 & 493500  \\
    \hline
		 JZ7W &  2000 & + &       & 9.4172e+02 & 6.7141e-02 & 497000  \\
    \hline 
  \end{tabular}
\end{table}
\normalsize
\end{frame}

\begin{frame}
\frametitle{$\pt$ spectra of Truth Jets}
  \begin{figure}[H]
    \centering
    \includegraphics[width=\textwidth]{{ptTruthAllRapidityBins}.eps}
  \end{figure}
\end{frame}

\subsection{Event Selection}

\begin{frame}
\frametitle{Event Selection}
\onslide<1-> \textbf{Remove the jets} with $\pt$ and rapidity $y$ out of used binning
\begin{itemize}

  \item \textbf{$\mathbf{\pt}$ Cut}
  
    Reco and truth jets with $\pt > 15 \GeV$ were kept.
  \item \textbf{$\mathbf{y}$ Cut}

    Reco and truth jets with $|y| < 4$ were kept.
\end{itemize}
\onslide<2-> \textbf{Remove the events} badly reconstructed by the detector
\begin{itemize}
  \item \textbf{Zero Jet (0-jet) Cut}
    
    Events with at least one reco and one truth jet, after the
    $\pt$ and $y$ cuts, are considered.
    
  \item \textbf{Leading Ratio (LR) Cut}

    If $0.6 < LR < 1.4$ the event is considered
    \begin{equation*}
      LR = p_{T,leading}^{reco} / p_{T,leading}^{truth} 
    \end{equation*}
\end{itemize}
\end{frame}

\begin{frame}
\frametitle{Event Selection - Truth Jets}
\begin{figure}[b]
  \centering
  \includegraphics[width=\textwidth]{{TruthCutting}.eps}
\end{figure}
\end{frame}

\begin{frame}
\frametitle{Event Selection - Reco Jets}
\begin{figure}[b]
  \centering
  \includegraphics[width=\textwidth]{{SignalCutting}.eps}
\end{figure}
\end{frame}

\subsection{Jet Matching}

\begin{frame}
\frametitle{Jet Matching}
\begin{itemize}
  \item In each event, for each truth jet, the corresponding reco jet has to be found.
  \item I have used \textit{\color{red}angular matching}
    \begin{enumerate}
      \item For each pair $(i,j)$ of reco and truth jets
        \begin{equation*}
          dR_{ij} = \sqrt{d\phi_{ij}^2 + dy_{ij}^2}
        \end{equation*}
      \item If $\min(dR_{ij}) = dR_{pq} < dR^{cutoff} = 0.2$ the jets (p,q) were
        matched and further not assumed
      \item Matching was done, when $\min(dR_{ij}) < dR^{cutoff}$ was not
        satisfied or all of the reco or truth jets were matched.
    \end{enumerate}
\end{itemize}
\end{frame}

\begin{frame}
\frametitle{Jet Matching - Truth Jets}
\begin{figure}[b]
  \centering
  \includegraphics[width=\textwidth]{{TruthMatching}.eps}
\end{figure}
\end{frame}

\begin{frame}
\frametitle{Jet Matching - Reco Jets}
\begin{figure}[b]
  \centering
  \includegraphics[width=\textwidth]{{SignalMatching}.eps}
\end{figure}
\end{frame}

\subsection{Unfolding}

\begin{frame}
\frametitle{Inputs for Unfolding}
Unfolding(calibrated reco spectrum) $\approx$ truth spectrum
\begin{itemize}
  \item Inputs for unfolding procedure 
    \begin{itemize}
      \item \textbf{Matching efficiencies} - describing the ratio of matched jets to all jets
      \item \textbf{Transfer matrix} $A_{ij}$ - containing the number of reco jets in bin $i$ with
        a matched truth jets generated in bin $j$
    \end{itemize}
  \item I test two approaches to the unfolding, allowing a dealing with the
    double binning (in $\pt$ and $y$)
\end{itemize}
\begin{enumerate}
  \item \textbf{Simple unfolding}

    Matching jets within different rapidity bins is not allowed. There are
    {\color{red}8~independent} 46x46 transfer matrices, one for each rapidity
    bin (46~=~number of $\pt$ bins)
  \item \textbf{2D unfolding}

    Matching within different rapidity bins allowed. {\color{red}Only one} 368x368 transfer
    matrix ($368=8 \times 48$)
\end{enumerate}
\end{frame}

\begin{frame}
\frametitle{Transfer Matrices}
\begin{columns}[onlytextwidth]
  \begin{column}{0.5\textwidth}
    \begin{figure}[H]
      \centering
    Simple unfolding
      \includegraphics[width=\textwidth]{{unfold_matrix_firstBin}.eps}
    \end{figure}
  \end{column}
  \begin{column}{0.5\textwidth}
    \begin{figure}[H]
      \centering
    2D unfolding
      \includegraphics[width=\textwidth]{{unfold_matrix_all}.eps}
    \end{figure}
  \end{column}
\end{columns}
\end{frame}

\begin{frame}
\frametitle{Slices in Transfer Matrix of 2D Unfolding}
\Put(265,20){\color{blue}\includegraphics[height=3cm]{{unfold_matrix_all}.eps}}
\begin{figure}[b]
  \raggedright
  \includegraphics[width=0.8\textwidth]{{UnfoldMatrixSlices11}.eps}
\end{figure}
\end{frame}

\begin{frame}
\frametitle{Matching Efficiencies}
\begin{columns}[onlytextwidth]
  \begin{column}{0.5\textwidth}
    \begin{figure}[H]
      \centering
    Truth jets
      \includegraphics[width=\textwidth]{{MatchEffSimpe2DTruth0Compare}.eps}
    \end{figure}
  \end{column}
  \begin{column}{0.5\textwidth}
    \begin{figure}[H]
      \centering
    Reco jets
      \includegraphics[width=\textwidth]{{MatchEffSimpe2DSignal0Compare}.eps}
    \end{figure}
  \end{column}
\end{columns}
\end{frame}

\begin{frame}
\frametitle{Steps of Unfolding}
\textbf{Three main steps of the unfolding procedure}
\begin{enumerate}
  \item Input data are multiplied by the matching efficiencies of reco jets
  \item Transfer matrix is used to correct data spectrum for detector effects. I
    use the \textit{Iterative Dynamical Stabilized} unfolding method with one iteration
  \item The spectrum obtained by the step 2 is divided by the matching
    efficiencies of truth jets, in order to correct resulting spectrum for the
    unmatched truth jets
\end{enumerate}
\end{frame}

\begin{frame}
\frametitle{Unfolding Results}
\begin{columns}[onlytextwidth]
  \begin{column}{0.5\textwidth}
    \begin{figure}[H]
      \centering
    Reco and Unfolded vs. Truth Spectrum
      \includegraphics[width=\textwidth]{{SignalUnfolded_VS_Truth0Compare}.eps}
    \end{figure}
  \end{column}
  \begin{column}{0.5\textwidth}
    \begin{figure}[H]
      \centering
    Simple and 2D unfolded vs. Truth Spectrum
      \includegraphics[width=\textwidth]{{UnfoldedSimpleComplex_VS_Truth0Compare}.eps}
    \end{figure}
  \end{column}
\end{columns}
\end{frame}

\section{NLO QCD Predictions}
\subsection{Introduction}

\begin{frame}
\frametitle{NLO QCD Prediction}
\begin{itemize}
  \item NLO QCD predictions on parton level for $\sqrt{s}=8\TeV$ and
    $\sqrt{s}=13\TeV$
  \onslide<1-> \item \textit{\color{red}Theoretical uncertainties} which are taken into account
  \begin{itemize}
    \item \textbf{Scale uncertainty}

      Choice of renormalization and factorization scales, including
      neglecting the higher order terms beyond the NLO
    \item \textbf{$\alpha_S$ uncertainty}

      Because of experimental measurements of $\alpha_S$.
    \item \textbf{PDF uncertainty}

      Prediction depends on the concrete choice of a PDF
  \end{itemize}
  \onslide<2-> \item \textit{\color{red}Other corrections} (not so significant)
  \begin{itemize}
    \item \textbf{Nonperturbative corrections}

      Hadronization and Underlying Event corrections.
    \item \textbf{Electroweak corrections}

      Next to the QCD processes, the electroweak processes should be assumed.
  \end{itemize}
\end{itemize}
\end{frame}

\subsection{Prediction Properties}

\begin{frame}
\frametitle{NLO Systematic Errors}
\begin{columns}[onlytextwidth]
  \begin{column}{0.5\textwidth}
    \begin{figure}[H]
      \centering
      $\sqrt{s}=8\TeV$
      \includegraphics[width=\textwidth]{{NLO_Systematics8_TeV0}.eps}
    \end{figure}
  \end{column}
  \begin{column}{0.5\textwidth}
    \begin{figure}[H]
      \centering
      $\sqrt{s}=13\TeV$
      \includegraphics[width=\textwidth]{{NLO_Systematics13_TeV0}.eps}
    \end{figure}
  \end{column}
\end{columns}
\end{frame}

\begin{frame}
\frametitle{Comparison of NLO QCD Predictions}
\begin{figure}[b]
  \centering
  \includegraphics[width=\textwidth]{{PredictionCompare0}.eps}
\end{figure}
\end{frame}

\subsection{Comparison with \textsc{Pythia8}}

\begin{frame}
\frametitle{Comparison of LO and NLO QCD}
\begin{figure}[b]
  \centering
  \includegraphics[width=0.7\textwidth]{{Truth_VS_Prediction0Compare}.eps}
\end{figure}
\end{frame}

\section{Conclusion}

\begin{frame}
\frametitle{Thesis Conclusions}
\begin{block}{Unfolding}
  Two approaches were probed.
  
  No significant differences between these two approaches imply, for the real
  analysis, the {\color{red}Simple Unfolding approach should be used} for its simpler
  implementation.

  Agreement of the unfolded $\pt$ spectra with the truth $\pt$ spectra up to
  systematic error $<10^{-3}\,\%$.
\end{block}
\begin{block}{LO and NLO QCD}
  Significant differences showing the influence of the NLO QCD processes on
  physical observables.
\end{block}
\end{frame}

\end{document} 
