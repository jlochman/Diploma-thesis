\chapter{QCD}
\label{ch:qcd}

\MyQuote{Is the purpose of theoretical physics to be no more than a cataloging of all the
things that can happen when particles interact with each other and separate? Or
is it to be an understanding at a deeper level in which there are things that
are not directly observable (as the underlying quantized fields are) but in
terms of which we shall have a more fundamental understanding?}
{Julian Schwinger}

The theoretical framework of particle physics is called the Standard Model. The
Standard Model describes the way how the fundamental components of matter
interact through strong, weak and electromagnetic interactions.
Mathematically, the
Standard Model is a gauge theory with local internal symmetries of
the direct product group $SU(3) \times SU(2) \times U(1)$. 
Gauge bosons, particles with integer spin, are assigned to generators of this
symmetry - there are 8 massless gluons from $SU(3)$ and 3 massive $W^\pm, Z$
bosons with 1 massless $\gamma$ boson from electroweak $SU(2) \times U(1)$
sector. 
In the electroweak sector, the Higgs Mechanism is introduced to assign $W^\pm, Z$
bosons masses and as a consequence, the new particle, Higgs boson, emerges in the
Standard Model. 

In addition to the bosons, the Standard Model introduces a spin-1/2 fermions,
which are divided into three quark and three lepton families. 
Fermions are assumed to be point-like, because there is no evidence for their
internal structure to date.
All fermions interact weakly, if they have electrical charge, they interact
electromagnetically as well. Quarks are the only fundamental fermions which
interact strongly. 
Figure~\ref{fig:SMparticles} shows the system of fundmamental particles of the
Standard Model.

\begin{figure}[t]
  \centering
  \includegraphics[width=\textwidth]{Chapter1/SM.png} 
  \caption[The system of fundamental particles of the SM.]
          {The system of fundamental particles of the SM. Figure taken from~
            \cite{wiki:SMParticlesSource}.} \label{fig:SMparticles}
\end{figure}

Quarks bind together to form hadrons and there are hundreds
\cite{PDG2014} of known hadrons up to date. Hadrons are
divided into baryons (3 quarks) and mesons (quark and anti-quark pairs). 
A theory called Quantum Chromodynamics (QCD) describes the strong interaction
between quarks.  
In this Chapter, I will discuss the key features of the QCD. 
I will give reasons for quark existence and for a decription of their strong
interaction as an $SU(3)$ gauge theory. 
After an introduction of a QCD Lagrangian, I will derive an expression for the
running coupling constant, which will be used to split the QCD itno perturbative
and non-perturbative regions. 
In these regions, the QCD has to describe the strong interaction with the use of
a different mathematical approaches. 
At the end of this Chapter, I will compare
these two approaches.

Most of the ideas, presented here, is overtaken from the textbook 
about the QCD \cite{QCDTextbook}. 
The electroweak sector of the Standard Model is described
in~\cite{horejsi2002fundamentals}. 
For more concise information about the Standard Model, the following textbooks
can serve \cite{griffiths2008introduction,cottingham2007introduction}.

\section{Theoretical Ansatz}
\label{Sec:TheoreticalAnsatz}

In 1950s, there had already been discovered tens of hadrons, thanks to new
particle accelerators, and a lot of effort was exerted to categorize them. 
Each hadron obtained a series of quantum numbers including isospin $T$ with
its third component $T_3$, hypercharge $Y$, electrical charge $Q$, strangeness
$S$, baryon number $B$ and others. 
Soon, people started to recognize some symmetries between these quantum numbers,
like the famous Gell-Mann--Nishijima relation
\cite{GellMannNishijima1,GellMannNishijima2}

\begin{equation}
  Q = T_3 + 1/2 Y \quad , \quad Y = B + S + \dots,
  \label{ex:GellMannNishijima}
\end{equation}
where dots denote charm, bottomness and topness, which were introduced after the
work of Gell-Mann and Nishijima. 
Some of the baryons, known in 1950s, are, together with their quantum numbers,
shown in Table~\ref{tab:SelectedHadrons}. 
In 1960s, the known hadrons were successfully categorized with the theory called
Eightfold Way, which was published independently by Murray Gell-Mann
\cite{Gell-Mann:101798} and George Zweig \cite{Zweig:570209} in 1964. 
The Eightfold Way successfully predicted the existence of a new particle
$\Omega^{-}$ including its mass. 
In this Section, I present a basic ideas of the Eightfold Way.

\begin{table}
  \centering
  \begin{tabular}{|C{1cm}|C{1cm}|C{1cm}|C{1cm}|C{1cm}|C{1cm}|}
    \hline
     & $S$ & $Y$ & $T$ & $T_3$ & $Q$  \\
    \hline \hline
    $p$ & \multirow{2}{*}{0} & \multirow{2}{*}{1} & \multirow{2}{*}{1/2} & 1/2  & 1 \\
    $n$ &                    &                    &                      & -1/2 & 0 \\
    \hline                                                              
    $\Sigma^+$  & \multirow{4}{*}{-1} & \multirow{4}{*}{0} & \multirow{3}{*}{1} & 1  & 1  \\
    $\Sigma^0$  &                     &                    &                    & 0  & 0  \\
    $\Sigma^-$  &                     &                    &                    & -1 & -1 \\
    $\Lambda$   &                     &                    & 0                  & 0  & 0  \\
    \hline                                                              
    $\Xi^0$ & \multirow{2}{*}{-2} & \multirow{2}{*}{-1} & \multirow{2}{*}{1/2} & 1/2 & 0  \\
    $\Xi^-$ &                     &                     &                      &-1/2 & -1 \\
    \hline
  \end{tabular}
  \caption{Quantum numbers of selected baryons known in 1950s. $S$ strangeness,
  $Y$ hypercharge, $T$ isospin, $T_3$ third component of isospin, $Q$ electrical
  charge.}
  \label{tab:SelectedHadrons}
\end{table}

The key feature of Eightfold Way is to understand hadron as a component of a
representation of infinitesimal generators of $SU(3)$ flavor symmetry
group. The infinitesimal generators of $SU(3)$ form a real eight-dimensional
Lie Algebra $\mathfrak{su}(3)$, which fundamental representation is usually
derived from Gell-Mann matrices

\begin{align}
  &\lambda_1 = \begin{pmatrix} 0 & 1 & 0 \\ 1 & 0 & 0 \\ 0 & 0 & 0 \end{pmatrix}
  \quad
  \lambda_2 = \begin{pmatrix} 0 & -i & 0 \\ i & 0 & 0 \\ 0 & 0 & 0 \end{pmatrix}
  \quad
  \lambda_3 = \begin{pmatrix} 1 & 0 & 0 \\ 0 & -1& 0 \\ 0 & 0 & 0 \end{pmatrix}
  , \nonumber \\
  &\lambda_4 = \begin{pmatrix} 0 & 0 & 1 \\ 0 & 0 & 0 \\ 1 & 0 & 0 \end{pmatrix}
  \quad
  \lambda_5 = \begin{pmatrix} 0 & 0 & -i\\ 0 & 0 & 0 \\ i & 0 & 0 \end{pmatrix}
  , \label{eq:GellMannMatrices} \\
  &\lambda_6 = \begin{pmatrix} 0 & 0 & 0 \\ 0 & 0 & 1 \\ 0 & 1 & 0 \end{pmatrix}
  \quad
  \lambda_7 = \begin{pmatrix} 0 & 0 & 0 \\ 0 & 0 & -i\\ 0 & i & 0 \end{pmatrix}
  \quad
  \lambda_8 = \frac{1}{\sqrt{3}} \begin{pmatrix} 1 & 0 & 0 \\ 0 & 1 & 0 \\ 
                                                              0 & 0 & -2 \end{pmatrix}.
  \nonumber
\end{align}

The generators, which are usually chosen as $g_a = \frac{1}{2} \lambda_a$, obey the
commutation relation $[g_a,g_b]=if_{abc}g_c$ with $f_{abc}$ being structure
constants. Cartan subalgebra of fundamental representation of $\mathfrak{su}(3)$
is generated by $H_1=g_3$ and $H_2=g_8$. The eigenstates of three-dimensional
representation of $\mathfrak{su}(3)$ can be chosen as
\small
\begin{equation}
  u = \begin{pmatrix} 1 \\ 0 \\ 0 \end{pmatrix} \leftrightarrow \left(
    \frac{1}{2}, \frac{\sqrt{3}}{6} \right), \quad
  d = \begin{pmatrix} 0 \\ 1 \\ 0 \end{pmatrix} \leftrightarrow \left(
    - \frac{1}{2}, \frac{\sqrt{3}}{6} \right), \quad
  s = \begin{pmatrix} 0 \\ 0 \\ 1 \end{pmatrix} \leftrightarrow \left(
    0, - \frac{\sqrt{3}}{3} \right), \quad
  \label{eq:RepresentLie3}
\end{equation}
\normalsize
where I have assigned the eigenvalues to generators of the Cartan subalgebra 
$H_1 u = \frac{1}{2} u$, $H_2 u = \frac{\sqrt{3}}{6} u$ and similarly
for $d$ and $s$ eigenstates. 
These eigenvalues are shown in Figure~\ref{fig:QuarkTriplet}. 

Next to the three-dimensional representation of $\mathfrak{su}(3)$, a
eight-dimensional adjoint representation can be defined.
This representation has the following eigenstates and eigenvalues

\begin{SCfigure}
  \centering
  \includegraphics[width=0.5\textwidth]{Chapter1/Quark-triplet.png} 
  \caption[Eigenvalues of three-dimensional representation of $\mathfrak{su}(3)$
          Lie Algebra.]
          {Eigenvalues of three-dimensional representation of $\mathfrak{su}(3)$ Lie
          Algebra. Figure taken from~\cite{LieAlgebrasForParticlePhysicists}.}
  \label{fig:QuarkTriplet}
\end{SCfigure}

\begin{align}
  \frac{1}{\sqrt{2}} \left( g_1 \pm i g_2  \right)
    &\leftrightarrow \left( \pm 1, 0 \right), \nonumber \\
  \frac{1}{\sqrt{2}} \left( g_4 \pm i g_5 \right) 
    &\leftrightarrow \left( \pm \frac{1}{2}, \pm \frac{\sqrt{3}}{2} \right), 
    \label{eq:RepresentLie8} \\
  \frac{1}{\sqrt{2}} \left( g_6 \pm i g_7 \right) 
    &\leftrightarrow \left( \mp \frac{1}{2}, \pm \frac{\sqrt{3}}{2} \right), \nonumber
\end{align}
where again, when denoting $A = \frac{1}{\sqrt{2}} ( g_1 + i g_2 )$, then the
upper sign of the first expression reads $[ H_1, A ] = A$, $[ H_2, A ] = 0$ and
similarly for remaining 5 eigenstates. Defining 

\begin{equation}
  H_1 = T_3 \quad \text{and} \quad H_2 = \frac{\sqrt{3}}{2} Y,
  \label{eq:LieIdentification}
\end{equation}
one can easily assign hadrons from Table~\ref{tab:SelectedHadrons} to
corresponding eigenvalues of the adjoint representation
in~\eqref{eq:RepresentLie8}, according to its third component of isospin $T_3$
and its hypercharge $Y$.  This is depicted in Figure~\ref{fig:BaryonicOctet}. 

When the same redefinition is done to the eigenstates of the three-dimensional
representation in~\eqref{eq:RepresentLie3}, one can assign to $u$, $d$, $s$
eigenstates the hypercharge $Y$ and strangeness $S$ as well. 
The concrete values for these states are shown in
Table~\ref{tab:SelectedQuarks}.

\begin{table}
  \centering
  \begin{tabular}{|C{1cm}|C{1cm}|C{1cm}|C{1cm}|C{1cm}|C{1cm}|}
    \hline
     & $S$ & $Y$ & $T$ & $T_3$ & $Q$  \\
    \hline \hline
    $u$ & \multirow{2}{*}{0} & \multirow{2}{*}{1/3} & \multirow{2}{*}{1/2} & 1/2
    & 2/3 \\
    $d$ &                    &                      &                      &
    -1/2 & \multirow{2}{*}{-1/3} \\
    $s$ & -1                 & -2/3                 & 0                    & 0    &  \\
    \hline                                                              
  \end{tabular}
  \caption{Quantum numbers of three quarks which existence was predicted by
    Gell-Mann and Zweig in 1964.}
  \label{tab:SelectedQuarks}
\end{table}

\begin{SCfigure}[][t]
  \centering
  \includegraphics[width=0.6\textwidth]{Chapter1/Baryon-octet.png} 
  \caption[Baryonic octuplet encapsulating baryons from
          Table~\ref{tab:SelectedHadrons}. For baryons in this diagram, the relation $Y
          = S + 1$ holds.]
          {Baryonic octuplet encapsulating baryons from
          Table~\ref{tab:SelectedHadrons}. For baryons in this diagram, the
          relation $Y = S + 1$ holds. Figure taken from \cite{wiki:EightFoldWay}.}
  \label{fig:BaryonicOctet}
\end{SCfigure}

Another representations of $\mathfrak{su}(3)$ Lie Algebra can be constructed.
The simplest way seems to be through the highest weight defining representation.
From eigenvalues of adjoint representation \eqref{eq:RepresentLie8} one can find
simple roots 
$\alpha^1=\left( \frac{1}{2}, \frac{\sqrt{3}}{2} \right)$, 
$\alpha^2=\left( \frac{1}{2}, - \frac{\sqrt{3}}{2} \right)$, 
from which the highest weights follow
$\mu^1=\left( \frac{1}{2}, \frac{\sqrt{3}}{6} \right)$, 
$\mu^2=\left( \frac{1}{2}, - \frac{\sqrt{3}}{6} \right)$. 
New representation of Lie Algebra can be constructed from the highest weights.
The whole procedure is described in \cite{LieAlgebrasForParticlePhysicists} in
detail.

Representations defined by the highest weight $\mu^1$ and $\mu^2$ respectively are
called fundamental. Fundamental representation defined by $\mu^1$ is usually
denoted $\mathbf{3}$ and we have encounter it already by the
expressions~\eqref{eq:RepresentLie3} and weight diagram in
Figure~\ref{fig:QuarkTriplet}. This representation corresponds to three
different quark states.  
The second fundamental representation, defined by the highest weight $\mu^2$,
corresponds to three anti-quark states and is usually denoted
$\bar{\mathbf{3}}$. 
The adjoint representation, depicted in Figure~\ref{fig:BaryonicOctet} is
defined by the highest weight $\mu^1 + \mu^2$.

Special interest is in representations with dimensions $10$ and $8$. These
are present in decompositions $\mathbf{3} \otimes \mathbf{3} \otimes
\mathbf{3} = \mathbf{10} \oplus \mathbf{8} \oplus \mathbf{8} \oplus \mathbf{1}$,
which corresponds to the baryons composed of three quarks, and $\mathbf{3}
\otimes \bar{\mathbf{3}} = \mathbf{8} \oplus \mathbf{1}$, corresponding to
mesons composed of quark and anti-quark.

Important feature of quark model just presented is its capability to predict
hadron masses. This is done using Gell-Mann--Okubo mass formula
\cite{Gell-Mann:1250016,Okubo01051962}

\begin{equation}
  M = a_0 + a_1 S + a_2 \left( T(T+1) - \frac{1}{4}S^2 \right),
  \label{eq:GellMannOkubo}
\end{equation}
where $a_0$, $a_1$ and $a_2$ are free parameters, which are common for all
hadrons in one multiplet. 

In 1970, Sheldon Lee Glashow, John Iliopoulos and Luciano Maiani proposed
\cite{Quarks4} an extension to the Eightfold Way, which predicted the existence of
a fourth flavor of quark - charm quark. 

In 1973, Makoto Kobayashi and Toshihide Moskawa proposed \cite{Quarks6}, that the
existence of 6 different quark flavors could explain the experimental
observation of CP violation.


\section{Experimental Ground}

In the previous Section, I have shown, that the hadrons can be categorized by
representations of $\mathfrak{su}(3)$ Lie Algebra. 
This lead to the model, where baryons were composed of three quarks and the
mesons of quark and anti-quark. 
In this Section, I summarize some experimental arguments to
support quark model.
Firstly, I will show, that the results from the lepton
scattering on nucleons can be explained by assumption, that nucleons are
composed of point-like spin-1/2 particles.
In the second part, I will encounter the question, why the group $SU(3)$ is
connected to the theory of strong interaction. 

\subsection{Scattering Reactions}

Inner structure of nucleon $N$ can be investigated by one of the following
scattering reactions

\begin{align}
  &e^- \, (E \gg 1\GeV) + N \rightarrow e^- + N,
  \label{eq:ScatteringReactionsElectron} \\
  &\nu_e \,\, (E \gg 1\GeV) + N \rightarrow \nu_e + N,
  \label{eq:ScatteringReactionsNeutrino}
\end{align}
where I have explicitly written $E \gg 1 \GeV$ to ensure, the wavelength
of lepton is $< 0.2\,\text{fm}$. By the first scattering reaction, the information
about the electric charge distribution in nucleon can be extracted, whereas the
second scattering reaction informs us about the weak charge distribution. 
From now on, I work only with scattering reaction
\eqref{eq:ScatteringReactionsElectron}, which was experimentaly examined as the
first. 
Feynmann diagram of this process is, together with kinematics variables and
vertex algebraic structures, depicted in Figure~\ref{fig:Scattering}. 

\begin{SCfigure}
  \centering
  \includegraphics[width=0.6\textwidth]{Chapter1/Scattering.png} 
  \caption[Scattering reaction $e^-N \rightarrow e^-N$ with kinematics variables
          and vertex algebraic structures.]
          {Scattering reaction $e^-N \rightarrow
          e^-N$ with kinematics variables and vertex algebraic structures.
          Figure taken from~\cite{QCDTextbook}.}
  \label{fig:Scattering}
\end{SCfigure}

Because of Lorentz-invariance of Quantum Electrodynamics, the matrix
element of the nucleon vertex $\bar{u}(P',S')\Gamma_\mu u(P,S)$ has to be a
Lorentz-vector. This restricts the possible form of $\Gamma_\mu$ to the
following algebraic structure

\begin{equation}
  \Gamma_\mu = A \gamma_\mu + B P_\mu' + C P_\mu + i D P'^\nu \sigma_{\mu\nu}
    + i E P^\nu \sigma_{\mu\nu},
  \label{eq:ScatteringAlgebraicMatrix}
\end{equation}
where $A$,\dots,$E$ depend only on Lorentz-invariant quantities. Next condition,
which has to be taken into account, is a gauge invariance of the matrix element, which
can be written in the form

\begin{equation}
  q^\mu \bar{u}(P',S')\Gamma_\mu u(P,S).
  \label{eq:ScatteringGaugeInvariance}
\end{equation}

The further computation of cross section is straightforward and the result can
be easily generalized to non-elastic scattering, by which the nucleon in final
state decays. The result is usually written using inelasticity parameter
$y=\frac{E-E'}{E}$, $0 \leq y \leq 1$, $y=0$ corresponding to the elastic
scattering, Bjorken variable $ x = \frac{Q^2}{2 P \cdot q}$, $ 0 < x \leq 1$, $x
= 1$ denoting elastic scattering and finally, instead of negative value $q^2$, the
$Q^2 = -q^2$ is used. Final result can be than written in the form

\begin{equation}
  \left. \frac{d^2\sigma}{dxdy} \right|_{eN} =
  \frac{8 \pi M_N E \alpha^2}{Q^4} \left[ x y^2 F_1^{eN}(Q^2, x)
  + (1-y) F_2^{eN}(Q^2,x) \right].
  \label{eq:ScatteringRes1}
\end{equation}
The $eN$ sub(super)script stresses the fact, we are dealing with scattering
\eqref{eq:ScatteringReactionsElectron}. 
$F_1^{eN}$ and $F_2^{eN}$, called Structure Functions, are not determinable by
the theory just presented - they have to be measured experimentally.

Structure Functions were first measured at $eP$ scattering, at SLAC in 1968
\cite{ePScattering}, and have shown the following results
\begin{enumerate}
  \item for $Q^2 \geq 1\GeV$, there is no significant dependence of Structure
    Functions on $Q^2$ and
  \item for $Q^2 \geq 1\GeV$, $F_2 \approx 2xF_1$.
\end{enumerate}
These results can be explained by assumption nucleon being composed of
point-like spin-1/2 constituents, for which R. P. Feynmann used the term
partons. In the following, I introduce the basic ideas of parton model. 

To $i$th parton, we assign momentum $P_{i,\mu}$

\begin{equation}
  P_{i,\mu} = \xi_i P_\mu + \Delta P_{i,\mu} 
    \quad , \quad \max_\mu (\Delta P_\mu) \ll \max_\mu P_\mu,
  \label{PartonsMomentumDistriburtionAssumption}
\end{equation}
where $\xi_i \in \left< 0, 1 \right>$ and $\Delta P_{i,\mu}$ comes from the
interaction between partons and we assume, the momentum coming from this
interaction is much smaller than the total nucleon momentum $P_\mu$. In
addition, probabilities $f_i(\xi_i)$ that $i$th parton will carry $\xi_i$
fraction of total momentum fulfilling

\begin{equation}
  \int d\xi_i f_i(\xi_i) = 1
  \label{eq:PartonDensityFunctionsNormalization}
\end{equation}
must be defined. Then, for the scattering reaction
\eqref{eq:ScatteringReactionsElectron} the formula for the cross section
can be derived

\begin{equation}
  \left. \frac{d^2\sigma}{dxdy} \right|_{eN} =
  \frac{4 \pi M_N E \alpha^2}{Q^4} \left[ y^2 + 2 ( 1 - y ) \right]
  \sum_i f_i(x) q_i^2 x,
  \label{eg:ScatteringRes2}
\end{equation}
where for $i$th parton its electrical charge $q_i$ was introduced. The last
expression and expression \eqref{eq:ScatteringRes1} can be compared as polynomials in $y$
resulting in

\begin{equation}
  F_1^{eN}(x) = \frac{1}{2} \sum_i f_i(x)q_i^2
  \quad , \quad
  F_2^{eN}(x) = \sum_i f_i(x) q_i^2 x.
  \label{eq:StructureFunctionAndPDF}
\end{equation}
It can be easily checked, that $F_2^{eN}(x) = 2 x F_1^{eN}(x)$. Functions
$f_i(x)$, just introduced, are called Parton Distribution Functions and their
important role in QCD will be discussed in
Section~\ref{sec:ComaprisonWithNLOPrediction} in more details.

The conclusion, which we should learn from this Section is, that the
experimental results of scattering reactions can be explained by assumption
nucleons being composed of spin-1/2 point-like partons, now called quarks. 

\subsection{Number of Colors}

Despite the strong confidence in the parton model, a theory, which would describe the
interaction between partons, was still missing. At the beginning of 1970s, there
was no direct evidence on how the theory would look like.
The theory of electroweak unification successfully suggested, that our Universe
at a subatomical level, could be described by a gauge theories, but 
to construct a gauge theory of strong interaction, the number of colors first had
to be known.

Number of colors $N_C$ is the number of different kinds of quarks of the same
flavor with respect to a new interaction. In this part, I present three arguments 
to demonstrate, that $N_C = 3$.

The first argument is the analysis of the electron-positron annihilation into the
pair of fermion and anti-fermion

\begin{equation}
  e^+e^- \rightarrow f\bar{f}.
  \label{ElectronPositronAnihilation}
\end{equation}

Feynmann diagram of this reaction is shown in Figure~\ref{fig:RRatio}, where
constants sitting in two vertices are emphasized.  $\alpha$ stands for Fine
Structure Constant and $Q_f$ for the charge of the fermion $f$ in units of positron
charge. The total cross section has to be proportional to

\begin{figure}[t]
  \centering
  \begin{subfigure}[b]{0.45\textwidth}
    \includegraphics[width=\textwidth]{Chapter1/RRatio.png} 
    \caption{$e^+e^- \rightarrow f\bar{f}$}
    \label{fig:RRatio}
  \end{subfigure}
  \quad
  \begin{subfigure}[b]{0.45\textwidth}
    \includegraphics[width=\textwidth]{Chapter1/PiMesonDecay.png}
    \caption{$\pi^0 \rightarrow 2 \gamma$}
    \label{fig:PiDecay}
  \end{subfigure}
  \caption{(a) $e^-e^+$ annihilation into the pair of fermion anti-fermion.
    Constants siting in both vertices are denoted, with $\alpha$ being the Fine
    Structure Constant and $Q_f$ the charge of fermion $f$ in units of positron
    charge.
           (b) $\pi^0$ meson decay into a pair of photons with closed fermion
         loop.}
  \label{fig:FeynmannGraphsNC3}
\end{figure}

\begin{equation}
  \sigma (e^- e^+ \rightarrow f \bar{f} ) \sim Q_f^2 \alpha^2.
  \label{eq:NumberOfColorsBasicCrossSection}
\end{equation}
In the case fermion $f$ being quark, there is new degeneracy coming from
different colors of quark-antiquark pair in final state, and the total cross
section has to be multiplied by factor $N_C$. Experimentally, the so called
$R$-factor is measured

\begin{equation}
  R = \frac{\sigma(e^+ e^- \rightarrow \text{hadrons})}{\sigma(e^+ e^-
  \rightarrow \mu^+ \mu^-)} = \left( \sum_q Q_q^2 \right) N_C,
  \label{eq:NumberOfColorsRatio}
\end{equation}
where the sum on the left hand side is over all possible quark flavors. When we
use the quark model proposed by Gell-Mann a Zweig, and substitute the values
from Table~\ref{tab:SelectedQuarks}, then

\begin{equation}
  R = \left[ \left( \frac{2}{3} \right)^2 +
    \left( \frac{-1}{3} \right)^2 +
  \left( \frac{-1}{3} \right)^2 \right] N_C = \frac{2}{3}N_C.
  \label{eq:NumberOfColorsSubstitued}
\end{equation}
Experimental results for $R$-ratio have shown \cite{PDG}, that $N_C = 3$.

The second argument, to support $N_C=3$, is the measurement of the decay width of
$\pi_0$ meson, which is depicted in Figure~\ref{fig:PiDecay}. For decay width
$\Gamma$ it can be derived 

\begin{equation}
  \Gamma = 7.63 \left( \frac{N_C}{3} \right)^2 \, \text{eV},
  \label{ex:PiMesonDecayWidth}
\end{equation}
which, compared to the experimental value $\Gamma = 7.57 \pm 0.32 \, \text{eV}$
\cite{PDG}, leads again to $N_C=3$.

The third argument is purely theoretical and states, that the Standard Model is
internally consistent only, if there are three colors \cite{QCDTextbook}. This
indicates, there is some linking between electroweak and strong sectors of
the Standard Model, and motivates the search for Grand Unified Theories.

\section{QCD as a Gauge Theory}

Putting arguments of the previous Section all together, there is the strong
experimental evidence, that nucleons consist of point-like spin-1/2 particles,
called quarks, and that quarks bring into the theory new degeneracy factor $N_C =
3$, which can be understood as three different strong charges. In
this Section, I follow the Yang-Mills theory \cite{YangMill} and define the QCD
Lagrangian.

Nowadays, the quark-quark strong interaction is understood as an $SU(3)$ gauge theory in
a degree of freedom called color. The generators of $SU(3)$ are derived from
Gell-Mann matrices \eqref{eq:GellMannMatrices} and act on quark color triplet
wave functions.

\begin{equation}
  \psi(x) = \begin{pmatrix}  
    \psi_r(x) \\ \psi_g(x) \\ \psi_b(x) \\ 
            \end{pmatrix}.
  \label{eq:QuarkWaveFunction}
\end{equation}
Following the Yang-Mills theory, to each generator
$\frac{\lambda^a}{2}$ a gluon field $A_\mu^a(x)$ and a gluon field strength tensor

\begin{equation}
  F_{\mu\nu}^a = \left( \partial_\mu A_\nu^a - \partial_\nu A_\mu^a + g f^{abc}
  A_\mu^b A_\nu^c \right)
  \label{eq:GluonFieldStrengthTensor}
\end{equation}
are assigned, where $g$ denotes the coupling constant of strong interaction and
$f^{abc}$ structure constant defined in Section~\ref{Sec:TheoreticalAnsatz}.
QCD Lagrangian

\begin{equation}
  \mathscr{L}_{\text{QCD}} = \bar{\psi} \left( -i \partial_\mu + g \frac{\lambda}{2}
  A_\mu^a(x) \right) \gamma^\mu \psi - \frac{1}{4}F_{\mu\nu}^aF_a^{\mu\nu}
  \label{eq:QCDLagrangian}
\end{equation}
is invariant under local transformation

\begin{align}
  &\psi(x) \, \, \, \rightarrow \psi'(x) = \Euler^{ig\Theta(x)} \psi(x),
    \label{eq:QCDGaugeTranform} \\
  &A_\mu(x) \rightarrow \Euler^{ig\Theta(x)} \left( A_\mu(x) +
    \frac{i}{g}\partial_\mu \right) \Euler^{-ig\Theta(x)}, 
  \nonumber
\end{align}
where

\begin{equation}
  \Theta(x) = \frac{1}{2} \lambda^a \Theta^a(x) 
  \quad , \quad
  A_\mu(x) = \frac{1}{2} \lambda^a A_\mu^a(x).
  \label{eq:QCDAdditionalFunctions}
\end{equation}

There is no mass term in Lagrangian \eqref{eq:QCDLagrangian}, because mass term
$m\bar{\psi}\psi$ vary under gauge transformation
\eqref{eq:QCDGaugeTranform}. To include quark mass term in QCD Lagrangian, the
Higgs mechanism \cite{HiggsMechanism}, which is explained in
\cite{horejsi2002fundamentals} in detail, has to be used.

QCD Lagrangian \eqref{eq:QCDLagrangian} together with gauge transformations
\eqref{eq:QCDGaugeTranform} are sufficient for determination of Feynman rules -
the key ingredient in the perturbative QCD, which we will, after one final
remark, discuss in the next Section.

By derivation of a gluon propagator, the gauge-fixing term has to be added to
the QCD Lagrangina.

\begin{equation}
  \mathscr{L}_{\text{QCD}}^{\text{gauge-fixing}} = - \frac{1}{2\xi} \left( \partial_\mu A_a^\mu
  \right)^2.
  \label{eq:QCDGaugeFixingTerm}
\end{equation}
This term confines the possible gauges to one class parametrized by real parameter
$\xi$. In non-Abelian gauge theories this term must be supplemented by the so
called ghost term, which brings into the theory a new unphysical scalar particle
obeying fermionic statistic. More details on the Faddev-Popov ghost field
can be found in~\cite{FaddeevPopovGhosts}.


\section{Perturbative QCD}

The Quantum Electrodynamics and The QCD are both the quantum field gauge theories, but
they differ in one killing feature - the former is Abelian whereas the latter is
not. The non-Abelian character of the QCD leads to new phenomenons of triple and
quartic gluonic interactions, which have an origin in the QCD Lagrangian
\eqref{eq:QCDLagrangian}. 
In this Section, I discuss one remarkable consequence - the running coupling
constant.

We start with the scattering process

\begin{equation}
  q \bar{q} \rightarrow q \bar{q},
  \label{eq:QuarkScattering}
\end{equation}
which is depicted in the lowest order of perturbation theory by the Feynman
graph in Figure~\ref{fig:QuarkQuarkScattering}. Except contribution of this
graph to the scattering amplitude (which is the only contribution $\sim g^2$)
there are 12 other Feynman diagrams with contributions $\sim g^4$. These are
depicted in Figure~\ref{fig:QuarkQuarkScatteringCorrection}. 

\begin{SCfigure}
  \centering
  \includegraphics[width=0.5\textwidth]{Chapter1/QuarkQuarkScattering.png} 
  \caption{Leading order Feynmann diagrams in the scattering reaction $q \bar{q}
    \rightarrow q \bar{q}$ with denoted transfered momentum $k$.}
  \label{fig:QuarkQuarkScattering}
\end{SCfigure}

\begin{figure}[t]
  \centering
  \includegraphics[width=\textwidth]{Chapter1/QuarkQuarkCorrection.png} 
  \caption{Perturbative corrections to the Feynmann diagram from
    Figure~\ref{fig:QuarkQuarkScattering} representing the scattering reaction
    $q \bar{q} \rightarrow q \bar{q}$. Dashed line represents a scalar ghost
    particle.}
  \label{fig:QuarkQuarkScatteringCorrection}
\end{figure}

The contributions from all the Feynman diagrams are calculated in~\cite{QCDTextbook}
in detail. There is shown, that all these contributions together are
logarithmically divergent. This divergence can be removed, when from the scattering
amplitude for arbitrary momentum transfer $k^2$ the scattering amplitude for fixed
momentum transfer $k^2 = -M^2$ is subtracted. This is how the renormalized
coupling constant $g_R$ is obtained and here is its final expression 

\begin{equation}
  g_R = g_0 - \frac{g_0^3}{16\pi^2} \left( \frac{11}{2} - \frac{1}{3}N_F \right)
  \ln \left( \frac{-k^2}{M^2} \right) + \mathscr{O}(g_0^5).
  \label{eq:RenormalizedCoupling}
\end{equation}
Here $g_0$ stands for the coupling constant measured at the renormalization scale
$k^2 = -M^2$ and $N_F$ for the number of different quark flavors with mass $m^2
\ll \left| k^2 \right|$. Dependence of $g_R$ on transfered momentum $k^2$ is
evident, but there are another two intertwined dependences - on renormalization
scale $M$ and on coupling constant at renormalization scale $g_0 =
\left. g_R \right|_{k^2=-M^2}$. For next purpose, it is convenient to use the
dependence schema

\begin{equation}
  g_R = g_R(-k^2,g_0(M)),
  \label{eq:RunningCouplingConstantDependenceSchema}
\end{equation}
which allows us to use the advantages of a $\beta$-function. With the usage of the
equation \eqref{eq:RenormalizedCoupling}, the differential equation for $g_0(M)$
can be obtained

\begin{align}
  \beta(g_0) \equiv M \left( \frac{\partial g_R}{\partial M} \right)_{-k^2=M^2}
  &= M \left( \frac{dg_0}{dM} \right)_{-k^2=M^2}
  \label{eq:BetaFunction1} \\
  &= -b_0 g_0^3 + \mathscr{O}(g_0^5)
  , \quad b_0 = \frac{1}{16\pi^2}\left(11-\frac{2N_F}{3}\right),
  \label{eq:BetaFunction2}
\end{align}
and solved directly to obtain coupling constant $g_0$ for arbitrary
scale $-k^2$

\begin{equation}
  \int_{g_0(M^2)}^{g_0(-k^2)} \frac{dg_0}{g_0^3} =
  -b_0 \int_{M^2}^{-k^2}\frac{dM}{M},
  \label{eq:RunningCouplingConstantIntegralEquation}
\end{equation}
with solution

\begin{equation}
  \alpha_S(-k^2) = \frac{\alpha_S(M^2)}{1 + \frac{\alpha_S(M^2)}{4\pi} \left(
  11-\frac{2N_F}{3} \right) \ln \left( \frac{-k^2}{M^2} \right) }
  , \quad g_0^2(-k^2) = 4 \pi \alpha_S( -k^2 ),
  \label{eq:RunningCouplingConstant}
\end{equation}
which is the final expression for running coupling constant up to one-loop
order. This dependence corresponds to experimental data, which are depicted in
Figure~\ref{fig:RunningCouplingConstant}. Coupling constant decreases with
increasing momentum transfer allowing the use of the perturbation theory. This
is known as the Principle of Asymptotic Freedom \cite{AssymptoticFreedom}.

\begin{figure}[t]
  \centering
  \includegraphics[width=0.8\textwidth]{Chapter1/RunningCouplingConstant.png}
  \caption[Experimental measurements of running coupling constant $\alpha_S(Q)$
          (solid line) and its uncertainty (yellow band).
          $Q=\sqrt{\left|k^2\right|}$ in comparison to
          \eqref{eq:RunningCouplingConstant}.]
          {Experimental measurements of running
          coupling constant $\alpha_S(Q)$ (solid line) and its uncertainty (yellow
          band).  $Q=\sqrt{\left|k^2\right|}$ in comparison to
          \eqref{eq:RunningCouplingConstant}. Figure taken from
          \cite{RunningCouplingConstantMess}. }
  \label{fig:RunningCouplingConstant}
\end{figure}

On the other hand, when the momentum transfer decreases, there is a special value
$-k^2=\Lambda^2$ for which the last expression diverges

\begin{equation}
  -1 = \frac{\alpha_S(M^2)}{4\pi} \left( 11 - \frac{2N_F}{3} \right)
  \ln \left( \frac{\Lambda^2}{M^2} \right).
  \label{eq:RunningLambda}
\end{equation}
Experimental value is $\Lambda=213^{+38}_{-35}\MeV$ \cite{wiki:QCDHistory} and
demonstrates, that the perturbative QCD cannot be used at low energy transfers.
In fact, the running coupling constant $\alpha_S(-k^2)$ reaches value $\sim 1$
on momenta transfers $\sqrt{\left| k^2 \right|} \sim 500\MeV$. 

The behavior of coupling constant at low energy transfers is not explainable in
the language of the perturpative QCD just presented. It is non-perturbative effect
known as the Principle of Color Confinment, which states, that quarks, when
seperate, the gluon force field between them becomes stronger instead of
diminishing. The accumulated energy is consumed by the creation of quark
anti-quark pair, until there is no free color charge left. This principle
forbids us from observing free quarks.

To understand e.g. structure of proton with rest mass $< 1\GeV$, it is clear, the 
non-perturbative QCD has to be used. Basic ideas of the non-perturbative QCD are
introduced in the next Section. 

\section{Non-Perturbative QCD}

The most well established non-perturbative approach to the QCD is a Lattice QCD.
In this Section, I discuss the basic features of the Lattice QCD. More
information on this extended topic can be found in~\cite{QCDTextbook,LQCDIntro}.

Lattice QCD is the QCD formulated on a hypercubic equally spaced lattice in
space and time, with a lattice parameter $a$ denoting the distance between
neighboring sites.
Quark fields are placed on sites, whereas the gluon fields sit on the links
between neighboring sites. From the QCD, the Lattice QCD inherits the gauge
invariance, which has to be formulated on the lattice structure.
For $a \rightarrow 0$ the Lattice QCD action coincides with that of QCD. 
The Lattice QCD contains 6 parameters - strong coupling constant and masses of 5
quarks (the top quark with lifetime $ \sim 10^{-24}\,\text{s}$ is not assumed by
the theory).

Unlike the perturbative expansions, used in perturbative QCD, the Lattice QCD
uses a numerical evaluation of a path integral to perform non-perturbative
calculations. 
Lattice QCD calculations are limited by the availability of computational
resources and the efficiency of algorithms. The Lattice QCD suffers with both
statistical and systematical errors, the former arising from the use of
Monte-Carlo integration, the latter, e.g. from the use of non-zero values of
$a$.

The current Lattice QCD calculations are made on supercomputers like the QCDCQ
supercomputer \cite{SuperComputer} with peak speed of 500 TFlops, using lattice
spacing $a \sim 0.05 - 0.15 \, \text{fm}$ in lattice volume $V \sim (2 - 6
\,\text{fm} )^3$.

The Importance of the Lattice QCD lies in its ability to predict masses of observed
mesons and baryons, including quark masses itself, and in investigation of
topological structure of a QCD vacuum.  The Lattice QCD can be used to obtain
Parton Distribution Functions
\eqref{eq:PartonDensityFunctionsNormalization}, helping us to understand the
structure of hadrons. Phenomenology of the Lattice QCD also explains the Principle of Color
Confinement. 


